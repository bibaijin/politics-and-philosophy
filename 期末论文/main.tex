\input /home/bibaijin/share/format/xetex.tex

\title{\FZLiuKai 论义利观与中国政治}
\author{{\FZLiuKai 张凯}
<\href{mailto:zhangk13@mails.tsinghua.edu.cn}{zhangk13@mails.tsinghua.edu.cn}>}

\begin{document}
\maketitle
\tableofcontents

\section*{}

\begin{quotation}

孟子见梁惠王。王曰:“叟!不远千里而来,亦将有以利吾国乎?”

孟子对曰:“王!何必曰利?亦有仁义而已矣。王曰:‘何以利吾国?’大夫曰:‘何以利吾
家?’士庶人曰:‘何以利吾身?’上下交征利而国危矣。万乘之国,弑其君者,必千乘之
家;千乘之国,弑其君者,必百乘之家。万取千焉,千取百焉,不为不多矣。苟为后义而
先利,不夺不餍。未有仁而遗其亲者也,未有义而后其君者也。王曰仁义而已矣,何必曰
利?”\parencite{wjwk-mz-lhws}

\end{quotation}

孟子的这一段话,可谓道尽了儒家的义利观。孔子曾说:“君子喻于义,小人喻于利。”,
其价值取向不言自明;荀子认为“义胜利者为治世,利克义者为乱世”;董仲舒说,“正其
谊不谋其利,明其道不计其功”;程颐说:“夫利,和义者善也,其害义者不善也”;王阳
明更因为“圣人之学日远日晦,而功利之习愈趣愈下”而痛心疾首……

那什么是义呢?《礼记·中庸》说:“义者,宜也。”即合适的意思。指公正合宜的道德、
道理或行为。韩愈认为“行而宜之之谓义”,即做事情顺应时宜,妥当处理。我认为,义是
指人们因为固有的善良本性而做出的符合道德规范的行为。一方面,义根源于人的善良天
性。孟子说:“羞恶之心,义之端也。”人们心中为自己的不正当行为感到羞愧,并且也憎
恶别人的不当行为。另一方面,人们会由于这种强烈的羞耻心,产生对公正和美德的追求
,做出符合道德规范的行为。义是一种行为。只有付诸实践,然后被社会大众认可,才能
称作是义。具体来说,尊敬兄长\footnote{《孟子·尽心上》:亲亲,仁也;敬长,义也
。}是义,对君主忠诚\footnote{《孟子·梁惠王上》:未有义而后其君者也。}是义,非
有勿取\footnote{《孟子·尽心上》:非其有而取之,非义也。}是义。义就是对道德行为
的追求。

而利则指物质利益。相比利益,儒家更重视仁义。如果人们不知道仁义礼仪,那么再多的
利益也满足不了人们的贪欲,因而弑君篡逆、互相攻伐的事情就会层出不穷。儒家希望推
行约束人们的礼仪规范、加强人们的道德修养,从而达到一个稳定而和谐的社会。儒家的
义利观深刻地影响了中国的政治生态和历史进程。

\section{义利观在历史上的成功典范}

对统治阶级来说,义利观有两层内涵:对外是反对侵略、保护弱国,维护国家间的公正秩
序,而不以大欺小、恃强凌弱;对内是为民制产、与民休息,而不与民争利,维护社会上
的公平正义。中国历史上辉煌的时期大抵都有一种高远的政治理想在背后支撑着。当政者
怀着济世安民之心励精图治,整个社会便会呈现出一片生机勃勃的景象。

\subsection{齐桓公的霸业}

齐桓公是春秋五霸之首。他率领诸侯在葵丘会盟的时候,周天子派宰孔赐给他祭庙用的胙
肉、彤弓矢以及天子车马,一时风光无限。孔子曾经说过:“微管仲,吾其被发左衽矣!”
可见,齐桓公的功绩有多么重要!当时周王室衰微,四方夷狄经常侵扰中原各国。如公元
前 663 年,山戎攻打燕国;公元前 661 年,狄人攻打邢国;公元前 658 年,狄人攻打
卫国。而燕、邢和卫等国弱小,无力抵抗狄人的进攻,有的甚至到了君死国灭的境地。而
齐桓公为救燕不惜跋涉千里,一直打到孤竹国为止;为邢卫建立新的离狄人较远的国都。
毫不夸张地说,正是齐桓公率领诸侯攘斥夷狄,华夏文明才能顺利地延续下来。

不仅攘夷,齐桓公还打出了尊王的旗号。公元前 656 年,齐桓公率诸侯联军大举伐楚,
责备楚国没有向周天子进贡包茅。公元前 655 年,齐桓公支持王太子郑成为周襄王。

齐国经管仲改革之后国力大盛,但并不以巧取豪夺、侵略欺骗称霸,而是靠尊王攘夷、匡
扶天下的政治主张赢得人心。葵丘会盟的盟书集中体现了齐桓公、管仲的政治理想:一是
诛杀不孝之人,不要改变已经确立的太子,不要让妾室成为正妻;二是尊重贤能之人,培
养人才,来表彰有德的人;三是尊敬长辈,爱护儿童,不怠慢来宾和旅客;四是士人不世
袭官职,官职不兼任,选用士人一定要得当,不能擅自杀戮大夫;五是不要筑堤防,不要
禁止邻国购买粮食,不要擅自封赏而不通告。\footnote{《孟子·告子下》:五霸,桓公
    为盛。葵丘之会诸侯,束牲、载书而不歃血。初命曰:“诛不孝,无易树子,无以妾
    为妻。”再命曰:“尊贤育才,以彰有德。”三命曰:“敬老慈幼,无忘宾旅。”四命曰
    :“士无世官,官事无摄,取士必得,无专杀大夫。”五命曰:“无曲防,无遏籴,无
    有封而不告。”曰:“凡我同盟之人,既盟之后,言归于好。”}

齐国称霸的过程中多次体现出了舍利取义的大度与高瞻远瞩。公元前 681 年,鲁国人曹
沫劫持桓公,要求归还鲁国失地。虽然桓公想反悔,但还是在管仲的劝谏下信守了诺言。
后来,还归还了卫国的台、原、姑和漆里四邑,归还燕国的柴夫和吠狗两邑。公元前663
年帮燕国打败山戎以后,燕庄公送齐桓公到齐境。桓公说:“不是天子的话,诸侯相送不
能处境,我不可以对燕无礼。于是把燕君所到的地方割给了燕国。\footnote{《史记·齐
    太公世家》:二十三年,山戎伐燕,燕告急于齐。齐桓公救燕,遂伐山戎,至于孤竹
    而还。燕庄公遂送桓公入齐境。桓公曰:“非天子,诸侯相送不出境,吾不可以无礼
于燕。”于是分沟割燕君所至与燕,命燕君复修召公之政,纳贡于周,如成康之时。诸侯
闻之,皆从齐。}公元前 659 年,桓公妹妹哀姜为患鲁国,桓公杀哀姜。公元前 656 年
,桓公率诸侯联军伐楚。虽然齐国带领的联军兵力强盛,但因为没有名正言顺的理由,在
责备楚国没有向周天子进贡包茅、要求楚国解释周昭王南征没有回来的原因之后就撤兵了
。信守在受胁迫情况下作出的承诺而不反悔,实力比别人强大的时候反而归还侵地,为了
维护宗法秩序而放弃领土,大义灭亲,因为没有正当的理由而停止进攻别的国家,每一件
事都是放弃了短期利益,却赢得了其他国家的信任与衷心拥戴。虽然传统上儒家对齐桓晋
文之事并不是特别推崇,而喜欢追述尧、舜、禹、汤、文、武、周公时候的事迹。但我认
为齐桓公、管仲的外交政策正是先义后利、舍利取义的典范。不论这种对义的追求是否出
自本心\footnote{桓公的好几次选择都是在管仲劝谏之下才改变主意的。比如因曹沫劫持
    而返还鲁国侵地的那次,《史记·刺客列传》记载:桓公与庄公既盟于坛上,曹沫执
    匕首劫齐桓公,桓公左右莫敢动,而问曰:“子将何欲?”曹沫曰:“齐强鲁弱,而大
    国侵鲁亦甚矣。今鲁城坏即压齐境,君其图之。”桓公乃许尽归鲁之侵地。既已言,
    曹沫投其匕首,下坛,北面就群臣之位,颜色不变,辞令如故。桓公怒,欲倍其约。
    管仲曰:“不可。夫贪小利以自快,弃信于诸侯,失天下之援,不如与之。于是桓公
    乃遂割鲁侵地,曹沫三战所亡地尽复予鲁。葵丘会盟的时候,桓公开始骄傲起来,差
    点对周天子无礼。《史记·齐太公世家》记载:三十五年夏,会诸侯于葵丘。周襄王
    使宰孔赐桓公文武胙、彤弓矢、大路,命无拜。桓公欲许之,管仲曰“不可”,乃下拜
    受赐。桓公晚年逐渐昏庸,好大喜功,甚至想僭越行天子之礼。他自认为“寡人南伐
    至召陵,望熊山;北伐山戎、离枝、孤竹;西伐大夏,涉流沙;束马悬车登太行,至
    卑耳山而还。诸侯莫违寡人。寡人兵车之会三,乘车之会六,九合诸侯,一匡天下。
    昔三代受命,有何以异于此乎?吾欲封泰山,禅梁父。”管仲固谏,不听;乃说桓公
以远方珍怪物至乃得封,桓公乃止。可见齐桓公尊王攘夷的出发点并不在于真的拥护周天
子,而是借王室名义赢得人心。},但他们的政策不仅赢得了众多小国的支持,震慑了强
国,达到了称霸天下的效果,而且客观上也维护了周朝宗法秩序的稳定。“威天下不以兵
革之利”,说的不就是桓公吗?

\subsection{唐代的租庸调制}

中国历史上最强盛的时期之一便是唐朝。唐太宗平东突厥、定薛延陀、灭高昌、并西域,
被西北各民族尊为“天可汗”;新罗、吐蕃、日本等国派遣大量人员赴唐学习,使中华文化
广播四方。而唐代之所以能取得如此杰出的成就,租庸调制功不可没。自古内政修明才能
招来远方,而唐代的租庸调制很好地贯彻了轻徭薄赋、为民制产的理念,为唐朝的文治武
功奠定了坚实的基础。

租庸调制需要均田制的配合。“凡男女始生为黄,四岁为小,十六为中,二十有一为丁,
六十为老。丁年十八以上授田一顷\footnote{五尺为步,二百四十步为亩,亩百为顷。}
内八十亩为口分,年老还官。二十亩为永业。”\parencite{qianmu}。这种均田制起源于
北魏,废除于晚唐。\parencite{wkpd-jtz}后来的农民起义军虽多次提出均田的口号
\footnote{如李自成提出了“均田免赋”\parencite{bdbk-jtmf}的口号,太平天国希望实
    现“有田同耕,有饭同食,有衣同穿,有钱同使,无处不均匀,无人不保暖”
    \parencite{wkpd-tctmzd}的理想社会。但一是因为战斗频繁,二是因为政策太理想
、不易执行,两者的政治口号并没有真正地大规模施行。},但从未如唐朝这般大规模、
有秩序地推行。按钱穆先生的观点,均田制的核心并不在于平均地权\footnote{“此田制
    并不在求田亩之绝对均给,只求富者稍有一限度,贫者亦有一最低之水准。”
    \parencite{qianmu}事实上,均田制是将官田分给百姓,并没有强征地主的私有土地
。},而在于为民制产。“及丁则授田,年老则还官”\parencite{qianmu},政府完全是从
农民的角度制定政策,百姓又怎能不安居乐业!

租庸调制的核心在于轻徭薄赋。“授田者丁岁输粟二石,谓之‘租’。丁随乡所出,岁输绫
、绢、絁各二丈,布加五之一。输绫、绢、絁者兼绵三两,输布者麻三斤,谓之‘调’。用
人之力,岁二十日,闰加五日。不役者日为绢三尺,谓之‘庸’。有事加役二十五日者,免
调;加役三十日者,租、调皆免。通正役不过五十日。”\parencite{qianmu}从地租来说
,当年孟子在战国时把十一之税奉为王者之政\parencite{}

\section{传统义利观在历史上遭遇的挫折}

\subsection{王莽篡汉和他的复古改制}

\subsection{东林党与明朝兴亡}

\subsection{中国在近代化转型中经历的曲折}

\section{义利观与中国的前途}

\subsection{由马基雅维里的《君主论》想到的}

拜读了马基雅维里的《君主论》之后,感慨万千。

分析成功与失败之原因。

+ 义只可律己,不可律人
+ 要顺应时势,而不是迂腐清高
+ 大义还是必要的

\subsection{儒家文化圈其他国家与地区的改革之路}

新加坡

韩国

台湾

\subsection{义利观的革新与发扬}

要有崇高的理想

水至清则无鱼

\printbibliography
\nocite{*}

\end{document}
