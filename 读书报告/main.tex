\input /home/bibaijin/.config/LaTeX/xelatex.ltx

\title{\FZLiuKai 读《君主论》有感}
\author{{\FZLiuKai 张凯}
<\href{mailto:zhangk13@mails.tsinghua.edu.cn}{zhangk13@mails.tsinghua.edu.cn}>}
\date{2015 年 11 月 15 日}

\begin{document}
\maketitle
\tableofcontents

\section*{}

尼科洛·马基雅维里深深地爱着自己的祖国。他生活在意大利城邦林立、屡受侵略的时代
:比希伯来人受奴役更甚,比波斯人更受压迫,比雅典人更加分散流离,既没有首领,也
没有秩序,受到打击,遭到劫掠,被分裂,被蹂躏,并且忍受了种种破坏。
\parencite{nikeluo}他热切的盼望着祖国能够日月重光,实现诗人佩脱拉克的话语:
\begin{quote}
  “反暴虐的力量,将拿起枪,\\
  战斗不会很长!\\
  因为古人的勇气,\\
  在意大利的心中至今没有消亡。”
\end{quote}

他写《君主论》的目的就在于希望为他的祖国开出救国的良方。马基雅维里的诚挚的爱国
心令我深受感动。而他在分析国运兴衰、权术智谋的时候,又是冷静的,甚至是冷酷的。
为了建立国家并长期保有,他不惜教导君主食言而肥、表里不一;他认为统帅的残酷无情
是必须的,否则汉尼拔就不可能带领由无数民族组成的大军在外国的土地上团结一致地作
战;他说一个人如果在一切事情上都想发誓以善良自持,那么,他侧身于许多不善良的人
当中定会遭到毁灭……虽然与我们平时的观念和受到的教育截然相反,但他的论断基于对人
性的深刻分析。他是第一个使政治学独立,同伦理学彻底分家的人。他的论断看似冷酷无
情,实则一针见血,直指人性的伪善;他的建议看似不择手段,实则极富远见,是为了给
国家带来长治久安。我认为,他的观点正是根源于他深厚的爱国心。因为爱得深沉,因为
祖国分崩离析、屡受欺凌,所以目光如炬、信念如铁。下面,我就谈一下自己的体会。

\section{国家的种类}

马基雅维里认为,一切国家,一切政权,不是共和国就是君主国。而君主国不是世袭的就
是新的。

\section{如何获得一个国家}

从布衣成为君主,需要能力或运气。而最不倚靠运气的人保持自己的地位最稳固。虽然依
靠能力成为君主的人在取得君权的时候是困难的,但以后保有它就容易了。因为取得君权
时发生的困难,部分由于为建立国家和长治久安,不得不采取新的规章制度。而采取新制
度是最困难、最危险的。因为革新者使所有在旧制度下过得顺利的人们都成了敌人,而那
些在新制度之下可能顺利的人们又仅仅是半心半意地拥护。这半心半意根源于人类的两个
心理特点——一是对他们的对手怀有恐惧,二是不轻易信任。人们对于新的事物在没有取得
可靠的经验以前,是不会真正相信的。这解释了为什么中国历史上的历次改革总是困难重
重。秦商鞅变法,废除贵族世袭的特权,奖励耕战,按军功大小受爵。所以贵族们反对他
,而老百姓怀疑他。他才不得不用徙木立信的方法争取信任。虽然变法取得了成功,而自
己却惨遭车裂之祸。吴起在楚悼王死后即被贵族射死,其变法也因此流产;王安石死前即
新法尽废,张居正身后家族蒙难、子孙罹祸,戊戌六君子喋血菜市口……自古以来守成法易
,而变法难,难道不是由于这个原因吗?赵武灵王胡服骑射、北魏孝文帝推行汉化,虽
功败垂成,而其远见与勇气,足为后世所效法。

那些光靠幸运,从平民崛起为君主的人们,在发迹时并不很艰难,但是要保持其地位,就
要殚精竭虑了。遽然勃兴的国家,如同自然界迅速滋生长大的其他一切东西一样,不能够
根深蒂固、枝桠交错,一旦遇到狂风暴雨就把它摧毁了。如吴王夫差之崛起,皆凭伍子胥
一人之谋,荐要离、孙武,既而平越、伐楚,因其成功太易,故其衰亡也速;如越王勾践
之复国,皆因夫差之有勇无谋,他自己虽然卧薪尝胆,不可谓不坚韧,但终因刻薄寡恩而
为楚所灭。尤其吴国,历次战役未尝败绩,而一次失败便足以亡国,难道不是因为它好大
喜功而很快失去了人民的支持吗?相比之下楚国虽历经劫难,但最终却吞并吴越之地,即
便后来被秦国灭亡,但“楚虽三户,亡秦必楚”的民间传说令人不寒而栗。为什么?因为楚
国在两湖地区经营数百年,楚怀王虽然因为天真而失去了国家,然而楚人却同情他,因此
陈胜、吴广起义的时候仍然不得不打着楚国的旗号。

以邪恶之道也可以获得君权。有些人屠杀市民、出卖朋友、缺乏信用、毫无恻隐之心、没
有宗教信仰,却能够长时期地在他们本国安全地生活下去,能够保卫自己不受外敌的侵害
,而且其本国公民也从没有阴谋反对他们。马基雅维里认为这是妥善地使用暴力使然:即
为了自己安全的必要,可以偶尔使用残暴手段,除非它能为臣民谋利益,其后决不再使用
。这是因为损害行为应该一下子干完,以便使人民少受一些损害,他们的积怨就少些;而
恩惠应该是一点儿一点儿地赐予,以便人民能更好地品尝恩惠的滋味。秦虽残暴,终于还
是灭了六国;而统一全国之后仍然按着以前的酷法,防范人民就像防范敌人一样,征戍劳
役繁重,以致二世即亡。

还有一种情况:不是依靠罪恶之道或者其他的凶暴行为,而是由于本土其他市民的赞助而
成为本国的君主,这种国家可以称之为市民的君主国。

\section{如何统治}

如果被征服的国家与征服的国家属于同一地区或使用同一语言,那么如果想保有它们,必
须注意两个方面:一方面是把他们的旧君的血统灭绝;另一方面是不要改变他们的法律和
赋税。而如果那些被征服的国家在语言、习惯和各种制度上同征服国不同,想保有他们便
会困难重重。最好和最有力的方法之一是征服者亲自前往、驻扎在那里。汉武帝以骑兵击
败匈奴后,又以步卒屯田,步步为营,终于使匈奴屈服\parencite{qianmu}。而东晋祖荻
、桓温、刘裕和南宋岳飞等多次北伐虽然前期收复了故土,但因为朝廷不愿北返,最终还
是功败垂成,胜利的果实又轻易失去。这不能不说是因为这个原因。另一个更好的对策是
在一两处战略要地殖民,否则必须在那里驻扎大批步兵和骑兵。因为殖民触犯的人较少,
而且他们贫困且散居各方,不能为害,所以是有益的;而驻军不得不把那个国家获得的全
部收入耗费掉,结果所得反而变成损失,每一个人都感到痛苦,于是都变成他的仇敌了,
因此是不中用的。

一个君主如果占有在语言、习惯和各种制度上同本国不同的地区,他就应该使自己成为那
些较弱小的邻近国家的首领和保护者,并且设法削弱它们当中较强大的势力,同时要注意
不让任何一个同自己一般强大的外国人利用任何意外事件插足那里。北宋联金灭辽,即是
养虎为患,遂有靖康之耻;齐桓公救燕、存卫、伐楚,实行的正是锄强扶弱的策略,因而
成为五霸之首。所以马基雅维里说,谁是促使他人强大的原因,谁就自取灭亡。

有的君主国由一位君主和一群臣仆统治,臣仆的地位来自君主的恩宠和钦许;有的君主国
由君主和诸侯统治,诸侯的权力来自世袭。占领前者是困难的,因为入侵者不可能由王国
的诸侯们召唤进来,也不能够指望依靠皇帝周围的人们叛变使其谋划获得便利。但是如果
一旦征服了皇帝,灭绝了君主的家族,入侵者就可以稳固地占有这个国家。比如大流士政
府是一个君主集权的国家,所以亚历山大把大流士彻底打垮之后就可以牢固地占有这个国
家。而对于一个由君主和诸侯共同统治的国家来说,占领它容易,长久地保有它却非常困
难。因为不同地方的人民会忠于当地的诸侯,而不会因为君主的倒台而轻易就支持侵略者
。

共和国的人民最难征服。因为他们向来习惯于在自己的法律下自由地生活。如果想保有这
种国家,有三种办法:其一是,把它们毁灭掉;其二是,亲自前往驻在那里;其三是,扶
植寡头政府。因为这种城市习惯于自由的生活,这种生活和秩序即使经过悠久的岁月或丰
厚的恩惠也不能够使人们忘怀。比如在佛罗伦萨人羁绊百年后的皮萨,人们遇到任何不测
之事就会立即想起以前的生活。

\section{君主应该遵守的行为规范}

马基雅维里认为,如果没有一些恶行,就难以挽救自己的国家的话,那么他也不必因为对
这些恶行的责备而感到不安。可见,他是从功利性的角度看待这一问题的。这也是马基雅
维里学说的一大特点——追求国家的统一与强盛,而不在乎出发点、手段与目的是否合乎道
德。

从个人来说,慷慨是美德。而君主却必须谨慎地对待它。因为君主的钱财或是自己的和老
百姓的,或是掠夺别人的。如果没有仔细考虑地任意花费钱财,他将不得不加重百姓的负
担,到那时人人都会怨恨他。作为君主必须眼光长远。

普通人一定觉得被人认为仁慈比被人认为残酷更好,而君主却应该反其道而行之。被人爱
戴与被人畏惧最好两者兼备,但如果要取舍的话,被人畏惧比被人爱戴安全得多。因为爱
戴是靠恩义维系的,而人性是恶劣的,在任何时候,只要对自己有利,人们便会把这条纽
带一刀两断。

虽然守信是令人称赞的品行,但往往是那些不重视守信、懂得怎样运用诡计的君主把那些
一贯守信的人们征服了。宋襄公死守春秋时代的交战规则,不肯半渡而击,因而被楚国大
败;秦国毫无顾忌地欺骗楚怀王,最终吞并楚国。君主需要在必要的时候果断实施阴谋诡
计。夫差刺杀吴王僚的行动谈不上光明正大,唐太宗的玄武门之变也不能说是迫不得已。
作为君主,就不能被通行的道德要求所限制。因为政治毕竟是人类社会中最残酷的竞争领
域,不站在制高点,就只能被淘汰。

\section{军队}

作者认为,一切国家,其主要的基础乃是良好的法律和良好的军队。军队有三种:他自己
的军队,雇佣军、援军或者混合的军队。雇佣军是靠不住的。作者认为意大利的崩溃即是
由于许多年来依赖雇佣军。因为雇佣军队是不团结的,怀有野心的,毫无纪律,不讲忠义
,在朋友前耀武扬威,在敌人面前则表现怯懦。唐朝在安史之乱中不得不向回纥借兵。但
请神容易送神难,回纥人后来屡次向唐室威逼勒索、劫掠边境甚至长安,唐朝也在委屈求
全中渐渐衰亡。因为只是为了雇佣的收入作战,雇佣军会采取一些有利于自己却未必有利
于雇佣国的策略。比如在意大利,雇佣军贬低步兵而重视骑兵,因为他们无法供养足够数
量的步兵。但是作为一支有战斗力的部队,他的兵种结构应该是平衡的。意大利的雇佣军
不夜袭城市,城市的防军也不夜袭野营;他们不在军营周围树立栅栏,也不挖掘壕沟;他
们不在冬季出征,等等。这些行动方式只是为了避免疲劳和危险而投机取巧。而一支部队
一旦有不敢做的事情而被别人揣摩透的话,就很容易被别人所利用了。所以作者强烈希望
建立自己的军队。

\section{结语}

马基雅维里是一位成熟的政治家。他平静地叙述着很多阴谋诡计而不义愤填膺,他深入地
分析这些阴谋家策略的得失成败而不纠缠于他们行为的崇高与否。他认为人性本恶,普通
民众经常在顺境时宣誓效忠君主,而一遇到风吹草动则很可能把君主抛弃。因而君主也应
该采取一切有利于自己统治的措施(当然君主还是应当尽力为善),而不必背负道德上的
负担。但他似乎完全是功利主义的。只要对占领、统治国家有利的事情,他就认为是正当
的、应该去做的,而没有明确地提出一种更高的治国理想。我认为一个国家想国运长久,
还是要有一种崇高的理想寄托在上面。纯粹的阴谋诡计会得到一时的成功,但最终会引来
大众普遍的怨恨。比如被作者奉为典范的博尔贾,其失败在作者看来是因为策略失误
\footnote{让自己损害过的人担任教皇},而我则认为带着某种必然性。从大的方面来说
是因为他过于依赖权术诡计,树敌太多;从细节来说,是因为想毒害政敌却误饮毒酒而染
病,这不能不说是一种讽刺。

\printbibliography
\nocite{*}

\end{document}
